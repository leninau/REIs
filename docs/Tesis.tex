\PassOptionsToPackage{unicode=true}{hyperref} % options for packages loaded elsewhere
\PassOptionsToPackage{hyphens}{url}
%
\documentclass[12pt,spanish,]{book}
\usepackage{lmodern}
\usepackage{amssymb,amsmath}
\usepackage{ifxetex,ifluatex}
\usepackage{fixltx2e} % provides \textsubscript
\ifnum 0\ifxetex 1\fi\ifluatex 1\fi=0 % if pdftex
  \usepackage[T1]{fontenc}
  \usepackage[utf8]{inputenc}
  \usepackage{textcomp} % provides euro and other symbols
\else % if luatex or xelatex
  \usepackage{unicode-math}
  \defaultfontfeatures{Ligatures=TeX,Scale=MatchLowercase}
\fi
% use upquote if available, for straight quotes in verbatim environments
\IfFileExists{upquote.sty}{\usepackage{upquote}}{}
% use microtype if available
\IfFileExists{microtype.sty}{%
\usepackage[]{microtype}
\UseMicrotypeSet[protrusion]{basicmath} % disable protrusion for tt fonts
}{}
\IfFileExists{parskip.sty}{%
\usepackage{parskip}
}{% else
\setlength{\parindent}{0pt}
\setlength{\parskip}{6pt plus 2pt minus 1pt}
}
\usepackage{hyperref}
\hypersetup{
            pdftitle={Géneración y Desarrollo de Recorridos de Estudio e Investigación en la Formación Matemática de Futuros Ingenieros},
            pdfauthor={Lenin Augusto Echavarría Cepeda},
            pdfborder={0 0 0},
            breaklinks=true}
\urlstyle{same}  % don't use monospace font for urls
\usepackage[margin=2cm]{geometry}
\usepackage{longtable,booktabs}
% Fix footnotes in tables (requires footnote package)
\IfFileExists{footnote.sty}{\usepackage{footnote}\makesavenoteenv{longtable}}{}
\usepackage{graphicx,grffile}
\makeatletter
\def\maxwidth{\ifdim\Gin@nat@width>\linewidth\linewidth\else\Gin@nat@width\fi}
\def\maxheight{\ifdim\Gin@nat@height>\textheight\textheight\else\Gin@nat@height\fi}
\makeatother
% Scale images if necessary, so that they will not overflow the page
% margins by default, and it is still possible to overwrite the defaults
% using explicit options in \includegraphics[width, height, ...]{}
\setkeys{Gin}{width=\maxwidth,height=\maxheight,keepaspectratio}
\setlength{\emergencystretch}{3em}  % prevent overfull lines
\providecommand{\tightlist}{%
  \setlength{\itemsep}{0pt}\setlength{\parskip}{0pt}}
\setcounter{secnumdepth}{5}
% Redefines (sub)paragraphs to behave more like sections
\ifx\paragraph\undefined\else
\let\oldparagraph\paragraph
\renewcommand{\paragraph}[1]{\oldparagraph{#1}\mbox{}}
\fi
\ifx\subparagraph\undefined\else
\let\oldsubparagraph\subparagraph
\renewcommand{\subparagraph}[1]{\oldsubparagraph{#1}\mbox{}}
\fi

% set default figure placement to htbp
\makeatletter
\def\fps@figure{htbp}
\makeatother

\usepackage{etoolbox}
\makeatletter
\providecommand{\subtitle}[1]{% add subtitle to \maketitle
  \apptocmd{\@title}{\par {\large #1 \par}}{}{}
}
\makeatother
\usepackage{booktabs}
\usepackage{setspace}
\doublespacing
% https://github.com/rstudio/rmarkdown/issues/337
\let\rmarkdownfootnote\footnote%
\def\footnote{\protect\rmarkdownfootnote}

% https://github.com/rstudio/rmarkdown/pull/252
\usepackage{titling}
\setlength{\droptitle}{-2em}

\pretitle{\vspace{\droptitle}\centering\huge}
\posttitle{\par}

\preauthor{\centering\large\emph}
\postauthor{\par}

\predate{\centering\large\emph}
\postdate{\par}
\ifnum 0\ifxetex 1\fi\ifluatex 1\fi=0 % if pdftex
  \usepackage[shorthands=off,main=spanish]{babel}
\else
  % load polyglossia as late as possible as it *could* call bidi if RTL lang (e.g. Hebrew or Arabic)
  \usepackage{polyglossia}
  \setmainlanguage[]{spanish}
\fi
\usepackage[]{natbib}
\bibliographystyle{plainnat}

\title{Géneración y Desarrollo de Recorridos de Estudio e Investigación en la Formación Matemática de Futuros Ingenieros}
\author{Lenin Augusto Echavarría Cepeda}
\date{01 de diciembre de 2019}

\begin{document}
\maketitle

{
\setcounter{tocdepth}{1}
\tableofcontents
}
\hypertarget{resumen}{%
\chapter*{Resumen}\label{resumen}}
\addcontentsline{toc}{chapter}{Resumen}

\hypertarget{glosario}{%
\chapter*{Glosario}\label{glosario}}
\addcontentsline{toc}{chapter}{Glosario}

\begin{description}
\item[Transposición]
Se da cuando un conocimiento va de un lugar a otro.
\item[Praxeología]
Unidad mínima de análisis.
\item[Conocimiento]
Elemento fundamental en las instituciones.
\item[Institución]
Organización social estable depositaria de conocimientos.
\item[Equipamiento praxeológico]
Conjunto de conocimientos que posee un individuo o una institución.
\end{description}

\hypertarget{teoruxeda-antropoluxf3gica-de-lo-diduxe1ctico}{%
\chapter{Teoría Antropológica de lo Didáctico}\label{teoruxeda-antropoluxf3gica-de-lo-diduxe1ctico}}

\hypertarget{aspectos-generales}{%
\section{Aspectos generales}\label{aspectos-generales}}

Conocimiento e institución son nociones estrechamente relacionadas dentro de la TAD. La existencia de las instituciones se justifica por los conocimientos que poseen y que pueden poner en práctica.

Dentro de la TAD, la unidad mínima de análisis de las actividades humanas es la praxeología.
Estas actividades pueden realizarse en cualquier ámbito donde intervienen las personas, como por ejemplo, el trabajo, la escuela, la investigación científica, la industria, etcétera.
El análisis empieza con la identificación de lo que se hace, la tarea o el tipo de tarea \(T\).
La descripción de cómo se hace se le conoce como la técnica \(\tau\).
Al discurso que justifica la técnica se le conoce como tecnología \(\theta\).
A su vez, la teoría \(\Theta\) es el discurso sobre la tecnología.
Estos cuatro elementos conforman la praxeología \([T,\tau,\theta,\Theta]\).
Así, una praxeología tiene dos bloques: la praxis \([T,\tau]\) y el logos \([\theta,\Theta]\).

Dentro de una institución, una praxeología \([T,\tau,\theta,\Theta]\) describe algo que alguno o algunos de sus mienbros son capaces de realizar.
Es decir, es una caracterízación de una parte del conocimiento que hay dentro de una institución. Varias praxeologías pueden compartir la misma tecnología \(\theta\).
Se dice así que varias praxeologías puntuales \([T,\tau,\theta,\Theta]\) se organizan en una praxeología local \([T_i,\tau_i,\theta,\Theta]\).
A su vez, las praxeologías locales se organizan en praxeologías regionales \([T_{ij},\tau_{ij},\theta_j,\Theta]\).
Siguiendo esta lógica, se puede pensar también en las praxeologías globales \([T_{ijk},\tau_{ijk},\theta_{jk},\Theta_k]\).

Los miembros de una institución son capaces de realizar tipos de tareas que son partes de praxeologías puntuales, locales, regionales o globales. Al conjunto de conocimientos conformados de esta manera relacionados con un individuo o una institución le llamamos equipamiento praxeológico.

La noción de transposición es uno de los más importantes en la teoría. Esta noción se hace necesaria cuando nos damos cuenta que los equipamientos praxeológicos son modificados cuando transitan de una institución a otra.

\hypertarget{niveles-de-codeterminaciuxf3n}{%
\subsection{Niveles de codeterminación}\label{niveles-de-codeterminaciuxf3n}}

\hypertarget{picm}{%
\section{PICM}\label{picm}}

\hypertarget{reis}{%
\subsection{REIs}\label{reis}}

\hypertarget{dialuxe9cticas}{%
\subsection{Dialécticas}\label{dialuxe9cticas}}

\hypertarget{actitudes}{%
\subsection{Actitudes}\label{actitudes}}

\hypertarget{mer}{%
\section{MER}\label{mer}}

\hypertarget{diseuxf1o-del-curso}{%
\chapter{Diseño del curso}\label{diseuxf1o-del-curso}}

\hypertarget{revisiuxf3n-general-de-la-disciplina}{%
\section{Revisión general de la disciplina}\label{revisiuxf3n-general-de-la-disciplina}}

\hypertarget{formulaciuxf3n-de-la-cuestiuxf3n-generatriz}{%
\section{Formulación de la cuestión generatriz}\label{formulaciuxf3n-de-la-cuestiuxf3n-generatriz}}

\hypertarget{desarrollo-del-rei}{%
\section{Desarrollo del REI}\label{desarrollo-del-rei}}

\hypertarget{resultados}{%
\chapter{Resultados}\label{resultados}}

\hypertarget{examen-de-la-disciplina}{%
\section{Examen de la disciplina}\label{examen-de-la-disciplina}}

\hypertarget{reportes-y-videos-de-las-cuestiones-generatrices}{%
\section{Reportes y videos de las cuestiones generatrices}\label{reportes-y-videos-de-las-cuestiones-generatrices}}

\hypertarget{reportes-y-presentaciones-finales}{%
\section{Reportes y presentaciones finales}\label{reportes-y-presentaciones-finales}}

\hypertarget{encuestas}{%
\section{Encuestas}\label{encuestas}}

\hypertarget{discusiuxf3n}{%
\chapter{Discusión}\label{discusiuxf3n}}

\hypertarget{dialuxe9cticas-1}{%
\section{Dialécticas}\label{dialuxe9cticas-1}}

\hypertarget{actitudes-1}{%
\section{Actitudes}\label{actitudes-1}}

\hypertarget{cognitivas}{%
\section{Cognitivas}\label{cognitivas}}

\hypertarget{hacia-un-nuevo-diseuxf1o}{%
\section{Hacia un nuevo diseño}\label{hacia-un-nuevo-diseuxf1o}}

\bibliography{Tesis.bib}

\end{document}
